\documentclass[10pt,letterpaper]{article}
\usepackage[top=0.85in,left=2.75in,footskip=0.75in]{geometry}

% Use adjustwidth environment to exceed column width (see example table in text)
%\usepackage{changepage}

% Use Unicode characters when possible
\usepackage[utf8]{inputenc}

% textcomp package and marvosym package for additional characters
%\usepackage{textcomp,marvosym}

% fixltx2e package for \textsubscript
\usepackage{fixltx2e}

% amsmath and amssymb packages, useful for mathematical formulas and symbols
\usepackage{amsmath,amssymb}

% cite package, to clean up citations in the main text. Do not remove.
%\usepackage{cite}

% Use nameref to cite supporting information files (see Supporting Information section for more info)
\usepackage{nameref}
\usepackage[colorlinks=true,urlcolor=blue]{hyperref}

% line numbers
%\usepackage[right]{lineno}

% ligatures disabled
\usepackage{microtype}
\DisableLigatures[f]{encoding = *, family = * }

% rotating package for sideways tables
%\usepackage{rotating}

% Remove comment for double spacing
%\usepackage{setspace} 
%\doublespacing

% Text layout
\raggedright
%\setlength{\parindent}{0.5cm}
%\textwidth 5.25in 
%\textheight 8.75in

% Bold the 'Figure #' in the caption and separate it from the title/caption with a period
% Captions will be left justified
\usepackage[aboveskip=1pt,labelfont=bf,labelsep=period,justification=raggedright,singlelinecheck=off]{caption}

% Use the PLoS provided BiBTeX style
\bibliographystyle{plos2015}

% Remove brackets from numbering in List of References
%\makeatletter
%\renewcommand{\@biblabel}[1]{\quad#1.}
%\makeatother

% Leave date blank
\date{}

% Header and Footer with logo
%\usepackage{lastpage,fancyhdr,graphicx}
%\usepackage{epstopdf}
%\pagestyle{myheadings}
%\pagestyle{fancy}
%\fancyhf{}
%\lhead{\includegraphics[width=2.0in]{PLOS-submission.eps}}
%\rfoot{\thepage/\pageref{LastPage}}
%\renewcommand{\footrule}{\hrule height 2pt \vspace{2mm}}
%\fancyheadoffset[L]{2.25in}
%\fancyfootoffset[L]{2.25in}
%\lfoot{\sf PLOS}

%% Include all macros below

%\newcommand{\lorem}{{\bf LOREM}}
%\newcommand{\ipsum}{{\bf IPSUM}}

%% END MACROS SECTION


%%% Begin BWP
%\usepackage{amsmath, amsthm, amssymb, wasysym, graphicx}
%\usepackage[small, hang, bf]{caption}
%\usepackage{natbib}
%\renewcommand\cite{\citep}
%\newcommand\citepossessive[1]{\citeauthor{#1}'s \citeyearpar{#1}}
\newcommand\eq[1]{Eq.~\ref{#1}}
\newcommand\fig[1]{Fig.~\ref{#1}}
\newcommand\sref[1]{Section~\ref{#1}}
\let\oldmarginpar\marginpar
\renewcommand{\marginpar}[1]{\oldmarginpar{\linespread{1}\scriptsize{#1}}}

% PLOS wants \paragraph for some reason...
%\renewcommand{\subsubsection}[1]{\paragraph{#1}}

\setlength{\marginparwidth}{55mm}


\newcommand\argmin{\mathop{\textrm{{\rm argmin}}}\limits}
\newcommand{\noprint}[1]{}


\hyphenation{cross-valid-ation}
%%% End BWP


\begin{document}
\appendix

\section{Online Resources}

\subsection{Data}
Data and training configuration files for the 8 birds discussed here are available from the Open Science Framework at DOI: 10.17605/OSF.IO/BX76R . DOIs for the versions of the programmes used here are provided for reproducibility; however, we generally recommend downloading the latest release.

\subsection{Song alignment}
Jeff Markowitz's song alignment software [11] can be found at
\url{https://github.com/jmarkow/zftftb}

\subsection{Training the neural network}
\begin{trivlist}
\item \url{https://github.com/gardner-lab/syllable-detector-learn}
\item DOI: 10.5281/zenodo.437555
\end{trivlist}

\begin{description}
\item[Installation:]\hfill
See {\tt README.md} in the repository for complete, up-to-date instructions.
\begin{itemize}
    \item Clone the Github repository onto your computer, and add it to your MATLAB path.
    \item You also need the following packages from MATLAB File Exchange in your MATLAB path:
      \begin{description}
      \item[Significant Figures] by Teck Por Lam.  File ID \#10669.
      \item[Generate maximally perceptually-distinct colors] by Tim Holy.  File ID \#29702.
      \end{description}
    \end{itemize}

\item[Usage:]\hfill
    \begin{itemize}
    \item The main programme is {\tt learn\_detector.m}.
    \item The top section of {\tt learn\_detector.m}
      describes the configuration variables.
    \item The current directory should contain the data in {\tt
      song.mat}. The format for this file is specified in {\tt
      README.md}: a MATLAB file containing aligned song, nonsong
      (calls, cage noise, etc), and the audio sampling frequency.
    \item {\tt learn\_detector} also looks in the current directory for
      a configuration file {\tt params.m}, described in {\tt
        README.md}.  If the file is empty, the programme will display
      an image showing the average of the song spectrograms (frequency
      vs.~time), tell the user how to define syllables of interest,
      and exit.  If {\tt times\_of\_interest\_ms} is defined, the
      programme will train the neural network.  Optionally, {\tt
        params.m} can also override any of the configuration
      parameters at the top of {\tt learn\_detector.m}.
    \item Configuration may also be controlled by specifying
      parameter-value pairs to the function call in the common MATLAB
      fashion; for example, {\tt
        learn\_detector('times\_of\_interest\_ms', [150 200])}.
    \item The software will produce a neural network file, {\tt
      detector\_bird\_times\_\dots.mat} (the ellipses signify other
      information encoded in the filename).  It will also produce the
      audio test file described in Methods.
    \end{itemize}
\end{description}

\subsection{Runtime}
  
\subsubsection{MATLAB}
    \begin{trivlist}
    \item \url{https://github.com/gardner-lab/syllable-detector-matlab}
    \item DOI: 10.5281/zenodo.437557
    \end{trivlist}
    \begin{description}
    \item[Installation:]\hfill
      \begin{itemize}
      \item Clone the Github repository onto your computer.
      \item Usage requires MATLAB with the Data Acquisition Toolbox and the 
      Parallel Computing Toolbox.
      \end{itemize}
    \item[Connection:]\hfill
      \begin{itemize}
      \item Either connect the microphone directly into the microphone jack or through a 
      preamp into the audio input jack. The current implementation only supports a single 
      channel of audio.
      \item To generate TTL signals via the audio output, use the headphone or audio output
      jack.
      \item To generate TTL signals via an Arduino, switch to the {\tt arduino} branch, 
      load the MATLAB Arduino IO sketch onto the Arduino (the sketch is available in the 
      Github repository, and provides a simple serial protocol for controlling pins). 
      Connect the Arduino to the computer via USB. The TTL signal will be available on
      pin 7.
      \end{itemize}
    \item[Usage:]\hfill
      \begin{itemize}
      \item Switch to the directory of the repository and run {\tt nndetector\_live}. You
      can optionally specify parameters as string/value pairs. When you run the command,
      you will be prompted to select an input device, output device and select a ``.mat''
      file containing the trained detector.
      \item Note that you may need to tune the input and output buffer based on computer 
      performance. Try decreasing the size of the buffers until you see warnings about buffer
      under/overruns, then increase the value slightly. We usually found 2-3ms worked reliably.
      To specify the buffer sizes, use the {\tt buffer\_size\_input} and 
      {\tt buffer\_size\_output} parameters.
      \item Information about the detector is logged to a file called ``detector\_status.log''.
      \end{itemize}
    \end{description}
    
\subsubsection{Swift}
    \begin{trivlist}
    \item \url{https://github.com/gardner-lab/syllable-detector-swift}
    \item DOI: 10.5281/zenodo.437559
    \end{trivlist}
    \begin{description}
    \item[Installation:]\hfill
      \begin{trivlist}
      \item Download the most recent release from the Github repository onto your computer.
      \end{trivlist}
    \item[Connection:]\hfill
      \begin{itemize}
      \item Either connect the microphone directly into the microphone jack or through a 
      preamp into the audio input jack. If the input jack supports multiple channels (e.g., 
      audio input jacks are usually stereo), you can connect multiple audio sources and run 
      different detectors on each channel simultaneously.
      \item To generate TTL signals via the audio output, use the headphone jack. The 
      audio output must have at least the same number of channels as the input.
      \item To generate TTL signals via an Arduino, load the MATLAB Arduino IO sketch 
      onto the Arduino (the sketch is available in the Github repository, and provides a 
      simple serial protocol for controlling pins). Connect the Arduino 
      to the computer via USB. The TTL signal for the first input will be available on 
      pin 7, the second input on pin 8, etc.
      \end{itemize}
    \item[Usage:]\hfill
      \begin{itemize}
      \item Use {\tt convert\_to\_text.m} to convert the Matlab output file to a format 
      easily readable by Swift. This will generate a text version of the neural network.
      \item Launch the application downloaded above. It will provide an 
      menu to select an audio input and either an audio output or an Arduino output (if 
      detected). Once you select an input and an output, a new window will show all audio 
      channels associated with the input. For each channel, select the text-encoded version of 
      the detector created in the previous step. Once configured, press run to begin 
      monitoring the inputs.
      \end{itemize}
    \item[Customisation:]\hfill
      \begin{itemize}
      \item You can clone the Github repository and use Xcode to edit the project, if you 
      are interested in extending or modifying the functionality. To hook into either 
      the raw input or output, look at the ``Processor.swift'' file. The 
      {\tt Processor} class acts as a coordinator, receiving incoming audio, passing that
      to the detector and generating appropriate output TTL signals.
      \end{itemize}
    \end{description}
    
\subsubsection{LabVIEW}
    \begin{trivlist}
    \item \url{https://github.com/gardner-lab/syllable-detector-runtime-labview}
    \item DOI: 10.5281/zenodo.437558 
    \end{trivlist}
    \begin{description}
    \item[Installation:]\hfill
      \begin{trivlist}
      \item Clone the Github repository onto your computer.
      \end{trivlist}
    \item[Connection:]\hfill
      \begin{itemize}
      \item The microphone should be connected, optionally through a preamplifier, to the data acquisition card, by default on Analog Input 0 (AI0).
      \item Outputs are on the Digital I/O (DIO) channels.  See the block diagram of {\tt nnfft.vi} for documentation, or just connect to DIO0.
      \end{itemize}
    \item[Usage:]\hfill
      \begin{itemize}
      \item Use LabVIEW to open {\tt nnfft.vi}.
        \item A file chooser is provided in the front panel of {\tt nnfft.vi}.  Use it to point to the network file
      produced by {\tt learn\_detector.m}, and use LabView's ``Run''
      button to begin detection.
      \end{itemize}
    \end{description}
  
\end{document}
